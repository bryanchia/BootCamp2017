\documentclass[letterpaper,12pt]{article}
\usepackage{array}
\usepackage{threeparttable}
\usepackage{geometry}
\geometry{letterpaper,tmargin=1in,bmargin=1in,lmargin=1.25in,rmargin=1.25in}
\usepackage{fancyhdr,lastpage}
\pagestyle{fancy}
\lhead{}
\chead{}
\rhead{}
\lfoot{}
\cfoot{}
\rfoot{\footnotesize\textsl{Page \thepage\ of \pageref{LastPage}}}
\renewcommand\headrulewidth{0pt}
\renewcommand\footrulewidth{0pt}
\usepackage[format=hang,font=normalsize,labelfont=bf]{caption}
\usepackage{listings}
\lstset{frame=single,
  language=Python,
  showstringspaces=false,
  columns=flexible,
  basicstyle={\small\ttfamily},
  numbers=none,
  breaklines=true,
  breakatwhitespace=true
  tabsize=3
}
\usepackage{amsmath}
\usepackage{amssymb}
\usepackage{amsthm}
\usepackage{harvard}
\usepackage{setspace}
\usepackage{float,color}
\usepackage[pdftex]{graphicx}
\usepackage{hyperref}
\hypersetup{colorlinks,linkcolor=red,urlcolor=blue}
\theoremstyle{definition}
\newtheorem{theorem}{Theorem}
\newtheorem{acknowledgement}[theorem]{Acknowledgement}
\newtheorem{algorithm}[theorem]{Algorithm}
\newtheorem{axiom}[theorem]{Axiom}
\newtheorem{case}[theorem]{Case}
\newtheorem{claim}[theorem]{Claim}
\newtheorem{conclusion}[theorem]{Conclusion}
\newtheorem{condition}[theorem]{Condition}
\newtheorem{conjecture}[theorem]{Conjecture}
\newtheorem{corollary}[theorem]{Corollary}
\newtheorem{criterion}[theorem]{Criterion}
\newtheorem{definition}[theorem]{Definition}
\newtheorem{derivation}{Derivation} % Number derivations on their own
\newtheorem{example}[theorem]{Example}
\newtheorem{exercise}[theorem]{Exercise}
\usepackage{enumitem}
\newtheorem{lemma}[theorem]{Lemma}
\newtheorem{notation}[theorem]{Notation}
\newtheorem{problem}[theorem]{Problem}
\newtheorem{proposition}{Proposition} % Number propositions on their own
\newtheorem{remark}[theorem]{Remark}
\newtheorem{solution}[theorem]{Solution}
\newtheorem{summary}[theorem]{Summary}
%\numberwithin{equation}{section}
\bibliographystyle{aer}
\newcommand\ve{\varepsilon}
\newcommand\boldline{\arrayrulewidth{1pt}\hline}


\begin{document}

\begin{flushleft}
  \textbf{\large{Problem Set \#6}} \\
  Math, Jorge Barro\\
  Bryan Chia
\end{flushleft}

\vspace{5mm}

\noindent\textbf{Problem 8.1}\\
\\
\includegraphics [height=2.5in, width=4.5in]{graph1.png}
\vspace{0.2in}

The optimizer for the problem is $x = 5.28571, y = 0.714286$ at $z = 23.5714$.\\

\noindent\textbf{Problem 8.2}\\
\\
\includegraphics [height=2.5in, width=4.5in]{graph2.png}
\vspace{0.2in}

The optimizer for the problem is $x = 6, y = 12$ at $z = 20$.\\

\includegraphics [height=2.5in, width=4.5in]{graph3.png}
\vspace{0.2in}

The optimizer for the problem is $x = 15, y = 12$ at $z = 132$.\\

\noindent\textbf{Problem 8.5}\\

(i)\\

$
\begin{array}{c c c c c}
z & =& & +4x& +6y \\
\cline{1-5}
w_1 & =&11 & +x& -y \\
w_2 & =&27 & -x& -y \\
w_3 & =&90 & -2x& -5y \\
\end{array}
$
\vspace{0.2in}
$
\Rightarrow
\begin{array}{c c c c c}
z & =& 108 & +2y& -4w_2 \\
\cline{1-5}
w_1 & =&38 & -2y& -w_2 \\
w_2 & =&27 & -y& -w_2 \\
w_3 & =&36 & -3y& +2w_2 \\

\end{array}
$
\vspace{0.2in}
$
\Rightarrow
\begin{array}{c c c c c}
z & =& 132 & +\frac{4}{3}w_2& - \frac{2}{3}w_3 \\
\cline{1-5}
w_1 & =&14 &-\frac{4}{3}w_2& + \frac{2}{3}w_3 \\
w_2 & =&15 & - \frac{2}{3}w_2& + \frac{1}{3}w_3 \\
w_3 & =&12 & +\frac{2}{3}w_2& - \frac{1}{3}w_3 \\
\end{array}\\
$

(ii)\\

$
\begin{array}{c c c c c}
z & =& & +3x_1& +6x_2 \\
\cline{1-5}
w_1 & =&15 & -x_1& -3x_2 \\
w_2 & =&18 & -2x_1& -3x_2 \\
w_3 & =&4 & -x_1& - x_2 \\
\end{array}
$
\vspace{0.2in}
$
\Rightarrow
\begin{array}{c c c c c}
z & =& 12 & +4x_2& -3w_3 \\
\cline{1-5}
w_1 & =&11 & -4x_2& +w_3 \\
w_2 & =&10 & -5x_2& -2w_3 \\
x_1 & =&4 & +x_2& - w_3 \\
\end{array}
$
\vspace{0.2in}
$
\Rightarrow
\begin{array}{c c c c c}
z & =& 20 & -\frac{4}{5}w_2& - \frac{7}{5}w_3 \\
\cline{1-5}
w_1 & =&3 &-\frac{16}{5}w_2& + \frac{3}{5}w_3 \\
x_2 & =&2 & - \frac{1}{5}w_2& + \frac{2}{5}w_3 \\
x_1 & =&6 & +\frac{1}{5}w_2& - \frac{7}{5}w_3 \\
\end{array}\\
$

\noindent\textbf{Problem 8.17}\\

(i)\\

$
\begin{array}{c c c c c c}
z & =&  & &  & -x_0\\
\cline{1-6}
w_1 & =&-8 & +4x_2& +x_2 & +x_0\\
w_2 & =&6 & +2x_1& -3x_2 &+x_0\\
x_1 & =&3 & -x_1& &+x_0 \\
\end{array}
$
\vspace{0.2in}
$
\Rightarrow
\begin{array}{c c c c c c}
z & =& -8  & +4x_1& +2x_2 & -w_1\\
\cline{1-6}
x_0 & =&8 & -4x_2& -2x_2 & +w_1\\
w_2 & =&14& -2x_1& -5x_2 &+w_1\\
w_3 & =&11 & -5x_1& -2x_2 &+w_1 \\
\end{array}
$

The problem is unbounded.\\

(ii) The problem is infeasible.\\

(iii) Solving, we get $ x = 0, y=2$ which gives us a optimal value of $z=2$.\\

$
\begin{array}{c c c c c c}
z & =&  & &  & -x_0\\
\cline{1-6}
w_1 & =&4 & & -x_2 & +x_0\\
w_2 & =&6 & +2x_1& -3x_2 &+x_0\\
\end{array}
$
\vspace{0.2in}
$
\Rightarrow
\begin{array}{c c c c c c}
z & =&  4& & -x_2 & -w_1\\
\cline{1-6}
x_0 & =&-4 & & +x_2 & -w_1\\
w_2 & =&2 & +2x_1& -2x_2 &+w_1\\
\end{array}
$
\\
\vspace{0.2in}
$
\Rightarrow
\begin{array}{c c c c c c}
z & =&  6& +2x_1& -3x_2 & -w_2\\
\cline{1-6}
x_0 & =&-2 & +2x_1& +x_2 & -w_2\\
w_1 & =&-2 & -2x_1& +2x_2 &+w_2\\
\end{array}
$

\noindent\textbf{Problem 8.13}\\

By proposition 8.32, we have that either the origin is feasible, meaning that $x=0$, or that all entires are negative, meaning that either the optimal point for the auxiliary problem is 0, which means again that $x=0$, or no basic variable is binding and thus the problem is unbounded. 

\noindent\textbf{Problem 8.17}\\

For the problem of minimizing $\{ c^Tx: Ax \leq b, x\geq 0 \}$, we get the dual problem to maximize $\{ b^Ty: A^Ty \geq c, y\geq 0 \}$. Thus the dual of the dual is the problem: $\{ -b^Ty: -A^Ty \geq -c, y\geq 0 \}$, which can be considered a primal problem if $-b  = c$ and $-A^T = A$.\\

\noindent\textbf{Problem 8.18}\\

For the primal problem, we get the solution $x_1 = \frac{5}{4}, x_2 = \frac{1}{2}$, and for the dual problem, we get the solution $y_1 = \frac{1}{4}, y_2 = 0, y_3 = \frac{1}{4}$. We check that for every $x_i and z_i$, $x_iz_i = 0$ and $y_jw_j = 0$ for all $y_j$ and $w_j$.

\end{document}